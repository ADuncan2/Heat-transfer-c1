\documentclass[11pt]{article}
\usepackage[utf8]{inputenc}
\usepackage{graphicx}
\graphicspath{ {images/} }
\usepackage{amsmath}
\usepackage{placeins}

\title{Coursework 1 – Transient Conduction}
\author{Adam Duncan}
\date{\today}

\begin{document}

\maketitle

\section{\emph{Part A: Using lumped capacitance}}
\subsection{Assumptions}
\begin{itemize}
	\item Internal temperature of the steel ball is uniform at any time t.
	\item No change in water temperature.
	\item No heat transfer by radiation.
	\item Material is standard carbon steel.
	\item Material properties are constant (taken at average temperature $T = 469 ^{o}C$).
\end{itemize}
\subsection{Properties}
\begin{table}[h]
	\centering
	\caption{Properties from problem}
	\begin{tabular}{lllll}
		Property & Value & Unit &  &  \\ \cline{1-3}
		Characteristic   length, L & 5 & cm &  &  \\
		Diameter, D & 10 & cm &  &  \\
		Temperature   of the water, $T_w$ & 38 & $^oC$ &  &  \\
		Initial   temperature of steel ball, $T_{s,1}$ & 900 & $^oC$ &  &  \\
		Final   temperature of steel ball, $T_{s,2}$ & 200 & $^oC$ &  &  \\
		Heat transfer   coefficient, h & 600 & $W/m^{2}K$ &  & 
	\end{tabular}
	\label{tab1}
\end{table}

\begin{table}[h]
	\centering
	\caption{Properties from literature}
	\begin{tabular}{llll}
		Property & Value at $T_{avg}$(469 $^{o}C$) & Unit & Source \\ \hline
		Specific heat   capacity, Cp & 552 & $J\cdot kg^{-1} K^{-1}$ & \cite{jean-marc_franssen_fire_2015} \\
		Density, $\rho$ & 7.8 x $10^3$ & $kg\cdot m^{-3}$ & \cite{bergman_fundamentals_2011} \\
		Conductivity, $k$ & 40 & $W\cdot m^{-1} K^{-1} $& \cite{jean-marc_franssen_fire_2015}
	\end{tabular}
	\label{tab2}
\end{table}
\FloatBarrier

The density of steel is assumed to be constant over the temperature range so the value in table \ref{tab2}, which is given at 300K, is assumed to be accurate. To confirm this assumption is acceptable the elongation was calculated using the ISO 834 standard equations\cite{jean-marc_franssen_fire_2015}. This showed the overall change in volume of the sphere was ~3\% over the full temperature range of the problem. As $V \propto \rho$ this change is low enough to be discounted and for the assumption to be justified.

\begin{table}[h]
	\centering
	\caption{Properties error analysis}
	\begin{tabular}{lllll}
		T [$^{o}C]$ & $C_{p}$ [$J \cdot kg^{-1}K^{-1}$] & $\Delta$\% from $T_{469}$ & k [$W \cdot m^{-1}K^{-1}$] & $\Delta$\% from $T_{469}$ \\ \hline
		900 & 650 & 56 & 27.3 & -29 \\
		469 & 416 & 0 & 38.4 & 0 \\
		38 & 454 & 9 & 52.7 & 37
	\end{tabular}
	\label{tab3}
\end{table}

Similarly the variation in specific heat capacity ($C_p$) and conductivity ($k$) were considered over the full temperature range as shown in Table \ref{tab3}. The value of $k$ varies from the $T_{mean}$ value by roughly $\pm 30\%$  but the $T_{mean}$ value is only $4\%$ different from the mean of values of $k$ at the two temperature extremes. This fact coupled with the information given in ISO 835 standard \cite{jean-marc_franssen_fire_2015} suggest a linear change in $h$, which makes the approximation useful. 

The variation in $C_p$ is less clear cut, with the value at $T_{mean}$ falling much closer to the value at $T_{38}$ than the value at $T_{900}$ as showing in Table \ref{tab3}. This variation is due to the non-linear variation of $C_p$ with temperature \cite{jean-marc_franssen_fire_2015}. Further investigation would be needed get an accurate estimate of the level of error this approximation introduces to the result but this falls beyond the scope of this problem set so the approximation is used as a best estimate.
\pagebreak
\subsection{Schematic}
\begin{figure}[!htbp]
	\centering
	\includegraphics[width=0.85\textwidth]{part_a_fig}
	\caption{Part A schematic at initial and final state.}
	\label{fig:schem_a}
\end{figure}
\FloatBarrier
\subsection{Analysis}

Energy balance for closed system gives the following equation.
\begin{equation}\label{eqn_1}
	\stackrel{.}{Q} = hA(T_{s}-T_{f}) = C_{p}\rho V \frac{dT_{c}}{dt}
\end{equation}

Where $\stackrel{.}{Q}$ is heat [$W$], $h$ is the heat transfer coefficient [$W/m^{2}K$], $A$ is the surface area between the ball and water [$m^{2}$], $T_{s}$ is the temperature of the steel ball [$^{o}C$], $T_{f}$ is the temperature of the water [$^{o}C$], $C_{p}$ is the specific heat capacity [$J/mK$], $\rho$ is the density of the steel ball[$kg/m^{3}$], $V$ is the volume of the steel ball [$m^3$] and $t$ is the time [$s$].
\newline

Rearranging (\ref{eqn_1}) to separate the variables gives.
\begin{equation}\label{key}
	\frac{1}{T_{s}-T_{f}} dT_{c} = \frac{hA}{C_{p}\rho V}dt
\end{equation}

Which integrates to give.
\begin{equation}\label{eqn_3}
	\ln{(\frac{T_{s1}-T_{f}}{T_{s2}-T_{f}})} =  \frac{hA}{C_{p}\rho V}(t_{2}-t_{1})
\end{equation}
Where $t_{i}$ and $T_{si}$ are the time [s] and temperature [$^oC$] at state $i$ receptively.

Rearranging (\ref{eqn_3}) to make $t_{2}$ the subject gives.
\begin{equation}\label{key}
	t_{2} = \frac{C_{p}\rho V}{hA}(\ln{(\frac{T_{s1}-T_{f}}{T_{s2}-T_{f}})})
\end{equation}

Substituting in the values for the variables given in Tables \ref{tab1} and \ref{tab2} gives the final value.
\boldmath
\begin{equation}\label{t1}
	t_2 = 205 s
\end{equation}
\unboldmath
Where $t_2$ is the time for the steel ball to reach a temperature of $200^{o}C$ under given assumptions.

\section{\emph{Part B: Lumped capacitance justification}}
The lumped capacitance method is only valid if the ratio of the conductive heat transfer to convective heat transfer is low. This ratio is known as the Biot number, $Bi$, and is given by.
\begin{equation}\label{eqn_biot}
	Bi = \frac{h \cdot L_{c}}{k}
\end{equation}
Where $h$ is convective coefficient [$W/m^{2}K$], $L_{c}$ is the characteristic length [m] and $k$ is the conductivity [$W/m \cdot K$]. 

Applying the values from Tables \ref{tab1} and \ref{tab2}, choosing to set $L_{c}=R$ and substituting into equation (\ref{eqn_biot}) gives:
\begin{equation}\label{biot_num}
	Bi = 0.7
\end{equation}

If $Bi > 0.1$ then the lumped capacitance method is no longer applicable as the assumptions made introduce non-trivial errors \cite{bergman_fundamentals_2011}. This means that the result in part A is likely inaccurate.

It is worth noting that the choice of $L_c$ is significant. It is common to select $L_c$ to be the maximum distance over which a temperature gradient would occur, as has been done above, but the method from the mathematical derivation uses $L_c = \frac{V}{A_s}$ which for a sphere gives $L_c = \frac{R}{3}$. This means the use of $L_c = R$ will tend to overestimate the value of $Bi$. In this case however, using $L_c = \frac{R}{3}$ gives $Bi = 0.25$ so the result can still be assumed to be inaccurate using the lumped capacitance method despite the overestimate.

\section{\emph{Part C: Transient conduction}}
\subsection{Time interval to cool centre of sphere to $200 ^{o}C$}
As the initial and final temperature of the sphere are specified as well as the temperature of the fluid which is assumed to be constant, the ratio below can be calculated.
\begin{equation}\label{ratio_1}
	\frac{T_{(0,t)} -T_{f}}{T_{i} - T_{f}}
\end{equation}
Using the values in Table \ref{tab1} the answer to (\ref{ratio_1}) is $0.19$. Taking the inverse of (\ref{biot_num}) gives $1.7$. These two values allow the Fourier number to be determined from the Heisler chart \cite{multidimensional-transient}, which is determined to be.
\begin{equation}\label{Fr_1}
	Fr = 1.2
\end{equation}
The Fourier number is defined below.
\begin{equation}\label{Fr_2}
	Fr = \frac{tk}{\rho C_{p} R^{2}}
\end{equation}

Using the result from (\ref{Fr_1}) and substituting in the values from Tables \ref{tab1} and \ref{tab2} into (\ref{Fr_2}) gives the final result time taken.
\boldmath
\begin{equation}\label{t2}
	t = \frac{F_{o}\rho C_{p} R^{2}}{k} = 257 s
\end{equation}
\unboldmath

Comparing this result and (\ref{t1}), the result gained from the lumped capacitance method, it can be seen the two are of a similar magnitude but (\ref{t2}) is larger than (\ref{t1}). This matches with expectations as the assumptions used in section 1 were found to be near the boundary so the error introduced by applying them would be significant but still give an answer of the same magnitude. Also the temperature gradient that was found to be important for this case in section 2 would take additional time to reach the centre point so the transient response being larger than the lumped capacitance result is intuitive.

\subsection{title}

\section{\emph{Part D: Non-infinite water bath}}

\section{\emph{Part E: Equilibrium temperature}}

\bibliographystyle{plain}
\bibliography{refs}
\end{document}

